%----------------------------------------------------------------------------------------
%	PACKAGES AND OTHER DOCUMENT CONFIGURATIONS
%----------------------------------------------------------------------------------------

\documentclass{article}

\usepackage{fancyhdr} % Required for custom headers
\usepackage{lastpage} % Required to determine the last page for the footer
\usepackage{extramarks} % Required for headers and footers
\usepackage[usenames,dvipsnames]{color} % Required for custom colors
\usepackage{pgf} % Required to insert pgf plots from matplotlib
\usepackage{graphicx} % Required to insert images
\usepackage{listings} % Required for insertion of code
\usepackage{courier} % Required for the courier font
\usepackage{lipsum} % Used for inserting dummy 'Lorem ipsum' text into the template
\usepackage[utf8]{inputenc}
\usepackage[ngerman]{babel}
\usepackage{multirow}

% Margins
\topmargin=-0.45in
\evensidemargin=0in
\oddsidemargin=0in
\textwidth=6.5in
\textheight=9.0in
\headsep=0.25in

\linespread{1.1} % Line spacing

% Set up the header and footer
\pagestyle{fancy}
%\lhead{\hmwkAuthorName} % Top left header
\chead{\hmwkClass\ : \hmwkTitle} % Top center head
\rhead{\firstxmark} % Top right header
\lfoot{\lastxmark} % Bottom left footer
\cfoot{} % Bottom center footer
\rfoot{Page\ \thepage\ of\ \protect\pageref{LastPage}} % Bottom right footer
\renewcommand\headrulewidth{0.4pt} % Size of the header rule
\renewcommand\footrulewidth{0.4pt} % Size of the footer rule

\setlength\parindent{0pt} % Removes all indentation from paragraphs

%----------------------------------------------------------------------------------------
%	CODE INCLUSION CONFIGURATION
%----------------------------------------------------------------------------------------

\definecolor{MyDarkGreen}{rgb}{0.0,0.4,0.0} % This is the color used for comments
\lstloadlanguages{Perl} % Load Perl syntax for listings, for a list of other languages supported see: ftp://ftp.tex.ac.uk/tex-archive/macros/latex/contrib/listings/listings.pdf
\lstset{language=Perl, % Use Perl in this example
        frame=single, % Single frame around code
        basicstyle=\small\ttfamily, % Use small true type font
        keywordstyle=[1]\color{Blue}\bf, % Perl functions bold and blue
        keywordstyle=[2]\color{Purple}, % Perl function arguments purple
        keywordstyle=[3]\color{Blue}\underbar, % Custom functions underlined and blue
        identifierstyle=, % Nothing special about identifiers                                         
        commentstyle=\usefont{T1}{pcr}{m}{sl}\color{MyDarkGreen}\small, % Comments small dark green courier font
        stringstyle=\color{Purple}, % Strings are purple
        showstringspaces=false, % Don't put marks in string spaces
        tabsize=5, % 5 spaces per tab
        %
        % Put standard Perl functions not included in the default language here
        morekeywords={rand},
        %
        % Put Perl function parameters here
        morekeywords=[2]{on, off, interp},
        %
        % Put user defined functions here
        morekeywords=[3]{test},
       	%
        morecomment=[l][\color{Blue}]{...}, % Line continuation (...) like blue comment
        numbers=left, % Line numbers on left
        firstnumber=1, % Line numbers start with line 1
        numberstyle=\tiny\color{Blue}, % Line numbers are blue and small
        stepnumber=5 % Line numbers go in steps of 5
}

% Creates a new command to include a perl script, the first parameter is the filename of the script (without .pl), the second parameter is the caption
\newcommand{\perlscript}[2]{
\begin{itemize}
\item[]\lstinputlisting[caption=#2,label=#1]{#1.pl}
\end{itemize}
}

%----------------------------------------------------------------------------------------
%	DOCUMENT STRUCTURE COMMANDS
%	Skip this unless you know what you're doing
%----------------------------------------------------------------------------------------

% Header and footer for when a page split occurs within a problem environment
\newcommand{\enterProblemHeader}[1]{
%\nobreak\extramarks{#1}{#1 continued on next page\ldots}\nobreak
%\nobreak\extramarks{#1 (continued)}{#1 continued on next page\ldots}\nobreak
}

% Header and footer for when a page split occurs between problem environments
\newcommand{\exitProblemHeader}[1]{
%\nobreak\extramarks{#1 (continued)}{#1 continued on next page\ldots}\nobreak
%\nobreak\extramarks{#1}{}\nobreak
}

\setcounter{secnumdepth}{0} % Removes default section numbers
\newcounter{homeworkProblemCounter} % Creates a counter to keep track of the number of problems

\newcommand{\homeworkProblemName}{}
\newenvironment{homeworkProblem}[1][Problem \arabic{homeworkProblemCounter}]{ % Makes a new environment called homeworkProblem which takes 1 argument (custom name) but the default is "Problem #"
\stepcounter{homeworkProblemCounter} % Increase counter for number of problems
\renewcommand{\homeworkProblemName}{#1} % Assign \homeworkProblemName the name of the problem
\section{\homeworkProblemName} % Make a section in the document with the custom problem count
%\enterProblemHeader{\homeworkProblemName} % Header and footer within the environment
}{
%\exitProblemHeader{\homeworkProblemName} % Header and footer after the environment
}

\newcommand{\problemAnswer}[1]{ % Defines the problem answer command with the content as the only argument
\noindent\framebox[\columnwidth][c]{\begin{minipage}{0.98\columnwidth}#1\end{minipage}} % Makes the box around the problem answer and puts the content inside
}

\newcommand{\homeworkSectionName}{}
\newenvironment{homeworkSection}[1]{ % New environment for sections within homework problems, takes 1 argument - the name of the section
\renewcommand{\homeworkSectionName}{#1} % Assign \homeworkSectionName to the name of the section from the environment argument
\subsection{\homeworkSectionName} % Make a subsection with the custom name of the subsection
%\enterProblemHeader{\homeworkProblemName\ [\homeworkSectionName]} % Header and footer within the environment
}{
%\enterProblemHeader{\homeworkProblemName} % Header and footer after the environment
}

%----------------------------------------------------------------------------------------
%	NAME AND CLASS SECTION
%----------------------------------------------------------------------------------------

\newcommand{\hmwkTitle}{Übung\ \#6} % Assignment title
\newcommand{\hmwkDueDate}{Montag,\ 01.\ Dezember\ 2014} % Due date
\newcommand{\hmwkClass}{Introduction to HPC} % Course/class
\newcommand{\hmwkClassTime}{} % Class/lecture time
\newcommand{\hmwkClassInstructor}{} % Teacher/lecturer
\newcommand{\hmwkAuthorName}{Günther Schindler, Christoph Klein, Klaus Naumann} % Your name

%----------------------------------------------------------------------------------------
%	TITLE PAGE
%----------------------------------------------------------------------------------------

\title{
\vspace{2in}
\textmd{\textbf{\hmwkClass:\ \hmwkTitle}}\\
\normalsize\vspace{0.1in}\small{Abgabe\ am\ \hmwkDueDate}\\
\vspace{0.1in}\large{\textit{\hmwkClassTime}}
\vspace{3in}
}

\author{\textbf{\hmwkAuthorName}}
\date{} % Insert date here if you want it to appear below your name

%----------------------------------------------------------------------------------------

\begin{document}

\maketitle

%----------------------------------------------------------------------------------------
%	TABLE OF CONTENTS
%----------------------------------------------------------------------------------------

%\setcounter{tocdepth}{1} % Uncomment this line if you don't want subsections listed in the ToC

\newpage
\tableofcontents
\newpage

%----------------------------------------------------------------------------------------
%	Heat Relaxation – Parallel Implementation
%----------------------------------------------------------------------------------------
%
%\begin{homeworkProblem}[Parallel implementation based on 1D-row partitioning]
%
%\end{homeworkProblem}
%\pagebreak
%----------------------------------------------------------------------------------------
%	Heat Relaxation - Experiments
%----------------------------------------------------------------------------------------
\begin{homeworkProblem}[Heat Relaxation II - Experiments]
Following table shows the average time for different grid-sizes. The measured results
base on 100 to 10.000 iterations, depending on the used grid-size.\\
\begin{center}
\begin{tabular}{ |c|c|c|c|c|c|c| }
\hline 
\textbf{Time / Iteration} & NP & NP & NP & NP & NP & NP\\
\hline
Grid Size & 2 & 4 & 6 & 8 & 10 & 12\\
\hline
128x128 & 0.000196 & 0.000217 & 0.000079 & 0.000065 & 0.000094 & 0.000100\\ 
\hline
512x512 & 0.003130 & 0.001246 & 0.001260 & 0.003804 & 0.001596 & 0.001323\\ 
\hline
1024x1024 & 0.013285 & 0.004686 & 0.005300 & 0.005174 & 0.006413 & 0.006468 \\ 
\hline
2kx2k & 0.052558 & 0.018573 & 0.021031 & 0.016017 & 0.025776 & 0.026442 \\
\hline
4kx4k & 0.227757 & 0.093330 & 0.066810 & 0.056120 & 0.046239 & 0.044635 \\
\hline
\end{tabular}
\end{center}
In order to calculate the Speedup we divided the execution time of the sequential implementation by the execution time of the parallel implementation for each grid-size and number of processes:
$$Speedup = \frac{T_{s}}{T_{p}}$$ \\
\begin{center}
\begin{tabular}{ |c|c|c|c|c|c|c|c| }
\hline 
\textbf{Speedup} & NP & NP & NP & NP & NP & NP & Sequential execution time\\
\hline
Grid Size & 2 & 4 & 6 & 8 & 10 & 12 &\\
\hline
128x128 & 0.311 & 0.281 & 0.772 & 0.938 & 0.649 & 0.610 & 0.000061\\ 
\hline
512x512 & 0.411 & 1.033 & 1.021 & 0.338 & 0.806 & 0.973 & 0.001287\\ 
\hline
1024x1024 & 0.549 & 1.555 & 1.375 & 1.408 & 1.136 & 1.127 & 0.007287\\ 
\hline
2kx2k & 0.563 & 1.592 & 1.406 & 1.846 & 1.147 & 1.118 & 0.029565\\
\hline
4kx4k & 0.501 & 1.224 & 1.71 & 2.035 & 2.470 & 2.559 & 0.114212\\
\hline
\end{tabular}
\end{center}
Here you can see that the maximum speedup is reached for a grid-size of \textit{4k x 4k} with 12 processes. At a grid-size of \textit{2k x 2k} we reached the maximum Speedup with 8 processes. With every additional processes the speedup for this grid-size decreases. The reason for this is the overhead of the parallel implementation. \\ 
This effect can be seen for the other grid-sizes as well which means that an additional amount of processes does not lead to an increasing speedup. \\
In order to calculate the efficiency we used the following formula
$$Efficiency = \frac{Speedup}{p}$$
in which p is the number of involved processes.\\
\begin{center}
\begin{tabular}{ |c|c|c|c|c|c|c| }
\hline 
\textbf{Efficiency} & NP & NP & NP & NP & NP & NP\\
\hline
Grid Size & 2 & 4 & 6 & 8 & 10 & 12 \\
\hline
128x128 & 0.156 & 0.070 & 0.129 & 0.117 & 0.0645 & 0.051 \\ 
\hline
512x512 & 0.206 & 0.258 & 0.170 & 0.042 & 0.081 & 0.081 \\ 
\hline
1024x1024 & 0.275 & 0.389 & 0.229 & 0.176 & 0.114 & 0.094 \\ 
\hline
2kx2k & 0.282 & 0.398 & 0.234 & 0.231 & 0.115 & 0.093 \\
\hline
4kx4k & 0.251 & 0.306 & 0.285 & 0.254 & 0.247 & 0.213 \\
\hline
\end{tabular}
\end{center}
This table represents the efficiency of our parallel implementation of the heat relaxation depending on grid-size and involved processes. For a grid-size of \textit{4k x 4k} with 12 involved processes - where we reached the highest speedup - we couldn't achieve the highest efficiency. For the given grid-size the maximum efficiency was reached with 4 involved processes. \\
For smaller grid-sizes the highest speedup and efficiency can be reached with the same amount of processes. \\ 
The bottom line is that the speedup and efficiency heavily depends on the problem size and the involved processes, whereas a higher amount of processes does not result in a faster and efficient computation. \\ \\ 
\begin{center}
%\includegraphics[width=0.6\columnwidth]{}
\end{center}

\end{homeworkProblem}
\clearpage
\end{document}
