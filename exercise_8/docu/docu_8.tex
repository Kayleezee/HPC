\documentclass[oneside, a4paper, DIV=10]{scrartcl}


% PACKAGES
\usepackage[english]{babel}
\usepackage{listings} % Required for insertion of code

% SETTINGS
\lstdefinestyle{customc}{
belowcaptionskip=1\baselineskip,
breaklines=true,
frame=L,
xleftmargin=\parindent,
language=C++,
numbers=left,
showstringspaces=false,
basicstyle=\footnotesize\ttfamily,
keywordstyle=\bfseries\color{green!40!black},
commentstyle=\itshape\color{purple!40!black},
identifierstyle=\color{blue},
stringstyle=\color{orange},
}
\lstset{style=customc}


% TITLE
\title{Exercise Sheet VIII}
\author{G\"unther Schindler, Klaus Naumann \& Christoph Klein}

% DOCUMENT
\begin{document}
\maketitle

% PART 1
%%%%%%%%
\section*{Reviews}
\begin{itemize}
    % FIRST REVIEW: MEMORY SENSITIVITY OF HPC APPLICATIONS
    \item
    The paper 'On the Effects of Memory Latency and Bandwidth on
    Supercomputer Application Performance' from Richard Murphy
    published in 2007 discusses the memory sensitivity of two types
    of high performance applications. There are traditional floating
    point based applications, which solves mostly physical problems,
    and emerging integer arithmetic based applications, which solve mostly
    graph problems.

    The author performed benchmarks for this two types of applications
    on a parallel machine simulator and measured the average instructions
    per clock cycle in dependency of the memory latency and bandwidth. Based
    on the results for the choosen benchmarks he says that both types of
    applications are rather latency than bandwidth sensitive. Furthermore
    the integer based applications were generally more memory 
    sensitive.

    To our mind this paper shows the divergence between HPC-architectures,
    which are optimized for traditional floating point operations, and the
    emerging amount of integer/tree based applications. But the author
    does not give constructive suggestions to solve this divergence. Furthermore
    one should keep in mind that the measurements were not done on a real HPC-system.
    This paper looks like an appeal at HPC-architects to consider the emerging
    amount of integer-based/memory-sensitive applications.

    % SECOND REVIEW: MEMORY ACCESS PATTERNS OF HPC APPLICATIONS
    \item text
\end{itemize}    
% PART 2
%%%%%%%%

% PART 3
%%%%%%%%
    
\end{document}
