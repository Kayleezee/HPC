%----------------------------------------------------------------------------------------
%	PACKAGES AND OTHER DOCUMENT CONFIGURATIONS
%----------------------------------------------------------------------------------------

\documentclass{article}

\usepackage{fancyhdr} % Required for custom headers
\usepackage{lastpage} % Required to determine the last page for the footer
\usepackage{extramarks} % Required for headers and footers
\usepackage[usenames,dvipsnames]{color} % Required for custom colors
\usepackage{pgf} % Required to insert pgf plots from matplotlib
\usepackage{graphicx} % Required to insert images
\usepackage{listings} % Required for insertion of code
\usepackage{courier} % Required for the courier font
\usepackage{lipsum} % Used for inserting dummy 'Lorem ipsum' text into the template
\usepackage[utf8]{inputenc}
\usepackage[ngerman]{babel}
\usepackage{multirow}

% Margins
\topmargin=-0.45in
\evensidemargin=0in
\oddsidemargin=0in
\textwidth=6.5in
\textheight=9.0in
\headsep=0.25in

\linespread{1.1} % Line spacing

% Set up the header and footer
\pagestyle{fancy}
%\lhead{\hmwkAuthorName} % Top left header
\chead{\hmwkClass\ : \hmwkTitle} % Top center head
\rhead{\firstxmark} % Top right header
\lfoot{\lastxmark} % Bottom left footer
\cfoot{} % Bottom center footer
\rfoot{Page\ \thepage\ of\ \protect\pageref{LastPage}} % Bottom right footer
\renewcommand\headrulewidth{0.4pt} % Size of the header rule
\renewcommand\footrulewidth{0.4pt} % Size of the footer rule

\setlength\parindent{0pt} % Removes all indentation from paragraphs

%----------------------------------------------------------------------------------------
%	CODE INCLUSION CONFIGURATION
%----------------------------------------------------------------------------------------

\definecolor{MyDarkGreen}{rgb}{0.0,0.4,0.0} % This is the color used for comments
\lstloadlanguages{Perl} % Load Perl syntax for listings, for a list of other languages supported see: ftp://ftp.tex.ac.uk/tex-archive/macros/latex/contrib/listings/listings.pdf
\lstset{language=Perl, % Use Perl in this example
        frame=single, % Single frame around code
        basicstyle=\small\ttfamily, % Use small true type font
        keywordstyle=[1]\color{Blue}\bf, % Perl functions bold and blue
        keywordstyle=[2]\color{Purple}, % Perl function arguments purple
        keywordstyle=[3]\color{Blue}\underbar, % Custom functions underlined and blue
        identifierstyle=, % Nothing special about identifiers                                         
        commentstyle=\usefont{T1}{pcr}{m}{sl}\color{MyDarkGreen}\small, % Comments small dark green courier font
        stringstyle=\color{Purple}, % Strings are purple
        showstringspaces=false, % Don't put marks in string spaces
        tabsize=5, % 5 spaces per tab
        %
        % Put standard Perl functions not included in the default language here
        morekeywords={rand},
        %
        % Put Perl function parameters here
        morekeywords=[2]{on, off, interp},
        %
        % Put user defined functions here
        morekeywords=[3]{test},
       	%
        morecomment=[l][\color{Blue}]{...}, % Line continuation (...) like blue comment
        numbers=left, % Line numbers on left
        firstnumber=1, % Line numbers start with line 1
        numberstyle=\tiny\color{Blue}, % Line numbers are blue and small
        stepnumber=5 % Line numbers go in steps of 5
}

% Creates a new command to include a perl script, the first parameter is the filename of the script (without .pl), the second parameter is the caption
\newcommand{\perlscript}[2]{
\begin{itemize}
\item[]\lstinputlisting[caption=#2,label=#1]{#1.pl}
\end{itemize}
}

%----------------------------------------------------------------------------------------
%	DOCUMENT STRUCTURE COMMANDS
%	Skip this unless you know what you're doing
%----------------------------------------------------------------------------------------

% Header and footer for when a page split occurs within a problem environment
\newcommand{\enterProblemHeader}[1]{
%\nobreak\extramarks{#1}{#1 continued on next page\ldots}\nobreak
%\nobreak\extramarks{#1 (continued)}{#1 continued on next page\ldots}\nobreak
}

% Header and footer for when a page split occurs between problem environments
\newcommand{\exitProblemHeader}[1]{
%\nobreak\extramarks{#1 (continued)}{#1 continued on next page\ldots}\nobreak
%\nobreak\extramarks{#1}{}\nobreak
}

\setcounter{secnumdepth}{0} % Removes default section numbers
\newcounter{homeworkProblemCounter} % Creates a counter to keep track of the number of problems

\newcommand{\homeworkProblemName}{}
\newenvironment{homeworkProblem}[1][Problem \arabic{homeworkProblemCounter}]{ % Makes a new environment called homeworkProblem which takes 1 argument (custom name) but the default is "Problem #"
\stepcounter{homeworkProblemCounter} % Increase counter for number of problems
\renewcommand{\homeworkProblemName}{#1} % Assign \homeworkProblemName the name of the problem
\section{\homeworkProblemName} % Make a section in the document with the custom problem count
%\enterProblemHeader{\homeworkProblemName} % Header and footer within the environment
}{
%\exitProblemHeader{\homeworkProblemName} % Header and footer after the environment
}

\newcommand{\problemAnswer}[1]{ % Defines the problem answer command with the content as the only argument
\noindent\framebox[\columnwidth][c]{\begin{minipage}{0.98\columnwidth}#1\end{minipage}} % Makes the box around the problem answer and puts the content inside
}

\newcommand{\homeworkSectionName}{}
\newenvironment{homeworkSection}[1]{ % New environment for sections within homework problems, takes 1 argument - the name of the section
\renewcommand{\homeworkSectionName}{#1} % Assign \homeworkSectionName to the name of the section from the environment argument
\subsection{\homeworkSectionName} % Make a subsection with the custom name of the subsection
%\enterProblemHeader{\homeworkProblemName\ [\homeworkSectionName]} % Header and footer within the environment
}{
%\enterProblemHeader{\homeworkProblemName} % Header and footer after the environment
}

%----------------------------------------------------------------------------------------
%	NAME AND CLASS SECTION
%----------------------------------------------------------------------------------------

\newcommand{\hmwkTitle}{Übung\ \#5} % Assignment title
\newcommand{\hmwkDueDate}{Montag,\ 24.\ November\ 2014} % Due date
\newcommand{\hmwkClass}{Introduction to HPC} % Course/class
\newcommand{\hmwkClassTime}{} % Class/lecture time
\newcommand{\hmwkClassInstructor}{} % Teacher/lecturer
\newcommand{\hmwkAuthorName}{Günther Schindler, Christoph Klein, Klaus Naumann} % Your name

%----------------------------------------------------------------------------------------
%	TITLE PAGE
%----------------------------------------------------------------------------------------

\title{
\vspace{2in}
\textmd{\textbf{\hmwkClass:\ \hmwkTitle}}\\
\normalsize\vspace{0.1in}\small{Abgabe\ am\ \hmwkDueDate}\\
\vspace{0.1in}\large{\textit{\hmwkClassTime}}
\vspace{3in}
}

\author{\textbf{\hmwkAuthorName}}
\date{} % Insert date here if you want it to appear below your name

%----------------------------------------------------------------------------------------

\begin{document}

\maketitle

%----------------------------------------------------------------------------------------
%	TABLE OF CONTENTS
%----------------------------------------------------------------------------------------

%\setcounter{tocdepth}{1} % Uncomment this line if you don't want subsections listed in the ToC

\newpage
\tableofcontents
\newpage

%----------------------------------------------------------------------------------------
%	Heat Relaxation – Sequential Implementation
%----------------------------------------------------------------------------------------

\begin{homeworkProblem}[Heat Relaxation – Sequential Implementation]
As part of the 5th exercise we should implement a program, which calculates the new 
value of grid points as average of the point itself and its four direct neighbors.
\\
Therefor we implemented the subroutine vRelaxMatrix(), which does the described
calculation. This function is based on the structure sMatrixDouble, used for the grid.
Before the function starts the calculation it copies the grid into a temporary grid.
Now the function calculates every new value of grid points, except the grid points
at the borders.
\begin{lstlisting}{c}
void vRelaxMatrix(sMatrixDouble *pM)
{
  int i,j;
  sMatrixDouble sMTmp;
  /* Allocate memory for temporary matrix */
  if(iAllocMatrixDouble(&sMTmp, pM->iRow, pM->iCol))
  {
    printf("DEBUG: Allocation failure!");
    exit(1);
  }
  /* Copy each matrix element in the temporary matrix */
  for(i=0; i<pM->iRow; i++)
    for(j=0; j<pM->iCol; j++)
      sMTmp.ppaMat[i][j]=pM->ppaMat[i][j];
  /* Calculate new grid points */
  for(i=0; i<pM->iRow; i++)
    for(j=0; j<pM->iCol; j++)
    {
      /* Inject 127.0 */
      if(i == 0 && j >= (pM->iCol/4) && j <= (3*(pM->iCol)/4))
        pM->ppaMat[i][j]=127.0;
      /* Set boundary points to 0.0 */
      else if(i==0  || i==pM->iRow-1 || j==0 || j==pM->iCol-1)
        pM->ppaMat[i][j]=0.0;
      else
      {
        pM->ppaMat[i][j]=sMTmp.ppaMat[i][j];
        pM->ppaMat[i][j]+=FI*((-4)*sMTmp.ppaMat[i][j] \
                        +sMTmp.ppaMat[i+1][j]+sMTmp.ppaMat[i-1][j] \
                        +sMTmp.ppaMat[i][j+1]+sMTmp.ppaMat[i][j-1]);
      }
    }
  /* Delete temporary matrix again */
  vFreeMatrixDouble(&sMTmp);
}
\end{lstlisting}
Also we should inject heat in the topmost grid points (j=0), with i in between
[N/4, 3N/4]. These grid points are set to 127. Every other grid point at the
borders is set to 0.0. This functionality is realised by the subroutine vInjectMatrix().
\begin{lstlisting}{C}
void vInjectMatrix(sMatrixDouble *pM)
{
  int i,j;
  
  for(i=0; i<pM->iRow; i++)
  {
    for(j=0; j<pM->iCol; j++)
    {
      if(i == 0 && j >= (pM->iCol/4) && j <= (3*(pM->iCol)/4))
	pM->ppaMat[i][j]=127.0;
      else
	pM->ppaMat[i][j]=0.0;
    }
  }
}
\end{lstlisting}
The main program now injects the heat with the function vInjectMatrix(). With every
iteration, it simulates the heat expansion as shown below.
\begin{lstlisting}{C}
   0.0 127.0 127.0 127.0 127.0 127.0   0.0
   0.0   0.0   0.0   0.0   0.0   0.0   0.0
   0.0   0.0   0.0   0.0   0.0   0.0   0.0
   0.0   0.0   0.0   0.0   0.0   0.0   0.0
   0.0   0.0   0.0   0.0   0.0   0.0   0.0
   0.0   0.0   0.0   0.0   0.0   0.0   0.0
   0.0   0.0   0.0   0.0   0.0   0.0   0.0

   0.0 127.0 127.0 127.0 127.0 127.0   0.0
   0.0  30.5  30.5  30.5  30.5  30.5   0.0
   0.0   0.0   0.0   0.0   0.0   0.0   0.0
   0.0   0.0   0.0   0.0   0.0   0.0   0.0
   0.0   0.0   0.0   0.0   0.0   0.0   0.0
   0.0   0.0   0.0   0.0   0.0   0.0   0.0
   0.0   0.0   0.0   0.0   0.0   0.0   0.0

   0.0 127.0 127.0 127.0 127.0 127.0   0.0
   0.0  39.0  46.3  46.3  46.3  39.0   0.0
   0.0   7.3   7.3   7.3   7.3   7.3   0.0
   0.0   0.0   0.0   0.0   0.0   0.0   0.0
   0.0   0.0   0.0   0.0   0.0   0.0   0.0
   0.0   0.0   0.0   0.0   0.0   0.0   0.0
   0.0   0.0   0.0   0.0   0.0   0.0   0.0
   
   0.0 127.0 127.0 127.0 127.0 127.0   0.0
   0.0  44.9  54.6  56.3  54.6  44.9   0.0
   0.0  11.4  14.9  14.9  14.9  11.4   0.0
   0.0   1.8   1.8   1.8   1.8   1.8   0.0
   0.0   0.0   0.0   0.0   0.0   0.0   0.0
   0.0   0.0   0.0   0.0   0.0   0.0   0.0
   0.0   0.0   0.0   0.0   0.0   0.0   0.0
   
   0.0 127.0 127.0 127.0 127.0 127.0   0.0
   0.0  48.1  60.5  62.5  60.5  48.1   0.0
   0.0  15.2  20.4  21.7  20.4  15.2   0.0
   0.0   3.2   4.5   4.5   4.5   3.2   0.0
   0.0   0.4   0.4   0.4   0.4   0.4   0.0
   0.0   0.0   0.0   0.0   0.0   0.0   0.0
   0.0   0.0   0.0   0.0   0.0   0.0   0.0
\end{lstlisting}
\end{homeworkProblem}
\pagebreak
%----------------------------------------------------------------------------------------
%	Heat Relaxation - Experiments
%----------------------------------------------------------------------------------------
\begin{homeworkProblem}[Heat Relaxation - Experiments]
\begin{center}
\begin{tabular}{ |c|c|c|c|c|c| } 
\hline
Grid Size & Time/iteration & Flops total & GFLOP/s \\
\hline
128x128 &  &  &  \\ 
\hline
512x512 &  &  &  \\ 
\hline
1024x1024 &  &  &  \\ 
\hline
2kx2k &  &  &  \\
\hline
4kx4k &  &  &  \\
\hline
\end{tabular}
\end{center}
\end{homeworkProblem}
%----------------------------------------------------------------------------------------
%	Heat Relaxation – Pre-considerations for parallelization
%----------------------------------------------------------------------------------------
\begin{homeworkProblem}[Heat Relaxation – Pre-considerations for parallelization]

\end{homeworkProblem} 
\clearpage
\end{document}
